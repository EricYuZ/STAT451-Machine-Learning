\documentclass[12pt]{article}
 
\usepackage[margin=1in]{geometry} 
\usepackage{amsmath,amsthm,amssymb}
 
\newcommand{\N}{\mathbb{N}}
\newcommand{\Z}{\mathbb{Z}}
 
\newenvironment{theorem}[2][Theorem]{\begin{trivlist}
\item[\hskip \labelsep {\bfseries #1}\hskip \labelsep {\bfseries #2.}]}{\end{trivlist}}
\newenvironment{lemma}[2][Lemma]{\begin{trivlist}
\item[\hskip \labelsep {\bfseries #1}\hskip \labelsep {\bfseries #2.}]}{\end{trivlist}}
\newenvironment{exercise}[2][Exercise]{\begin{trivlist}
\item[\hskip \labelsep {\bfseries #1}\hskip \labelsep {\bfseries #2.}]}{\end{trivlist}}
\newenvironment{problem}[2][Problem]{\begin{trivlist}
\item[\hskip \labelsep {\bfseries #1}\hskip \labelsep {\bfseries #2.}]}{\end{trivlist}}
\newenvironment{question}[2][Question]{\begin{trivlist}
\item[\hskip \labelsep {\bfseries #1}\hskip \labelsep {\bfseries #2.}]}{\end{trivlist}}
\newenvironment{corollary}[2][Corollary]{\begin{trivlist}
\item[\hskip \labelsep {\bfseries #1}\hskip \labelsep {\bfseries #2.}]}{\end{trivlist}}


\begin{document}

\title{Weekly Homework 1}
\author{Zeyang Yu\\
Math 421}
\date{February 2022}

\maketitle

\noindent On this week's homework, I finished it individually. I used the resources from class notes and textbook page 11 and 12. 

\begin{theorem}{1} 
If $m$ is an odd integer, then $m^2 + 5m + 12$ is even .
\end{theorem}

\begin{proof}
Since $m$ is odd, there is a integer $k$ such that $m = 2k + 1$.
\\$ m^2 + 5m + 12 
\\= (2k + 1)^2 + 5(2k + 1) + 12$
\\$=(4k^2 + 4k + 1) + 10k + 5 + 12$
\\$=4k^2 + 14k + 18$
\\$=2(2k^2 + 7k + 9)$
\\Thus, $m^2 + 5m + 12$ is divisible by 2, which means it is even. 
\end{proof}

\begin{theorem}{2} 
$|x \cdot y| = |x| \cdot |y|$
\end{theorem}
\begin{proof}
We will consider 4 cases:
\\(1) $x \geq 0, y \geq 0$
\\(2) $x \geq 0, y \leq 0$
\\(3) $x \leq 0, y \geq 0$
\\(4) $x \leq 0, y \leq 0$

In case (1) we have $x \cdot y \geq 0 $, then
\\$|x \cdot y| = x \cdot y = |x| \cdot |y|$,
\\so that in this case equality holds. 
\\In case (4) we have $x \cdot y \geq 0 $, and equality holds:
\\$|x \cdot y| = x \cdot y = |x| \cdot |y|$.
\\In case(2) we have $x \geq 0, y \leq 0$, and at this time, $x \cdot y \leq 0 $,
\\$|x \cdot y| = -(x \cdot y) = x \cdot (-y) = |x| \cdot |y|$, equality holds.
\\In case(3) we have $x \leq 0, y \geq 0$, and at this time, $x \cdot y \leq 0 $,
\\$|x \cdot y| = -(x \cdot y) = (-x) \cdot y = |x| \cdot |y|$, equality holds.
\\Thus, $|x \cdot y| = |x| \cdot |y|$ holds.
\end{proof}
\end{document}
